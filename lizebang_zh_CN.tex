% !TEX TS-program = xelatex
% !TEX encoding = UTF-8 Unicode
% !Mode:: "TeX:UTF-8"

\documentclass{resume}
\usepackage{zh_CN-Adobefonts_external}
\usepackage{linespacing_fix}
\usepackage{cite}

\begin{document}
\pagenumbering{gobble}

%%%%%%%%%%%%%%%%%%%%%%%%%%%%%%%%%%%%%%%%%%%%%%%%%%%%%%%%%%%%%%%%%%%%%%%%%%%%%%%%
\sffamily{\Huge 李泽邦}  \hfill
\vspace{2mm} \par
\textbf{应聘岗位:后台开发工程师}
\vspace{2mm} \par
\textbf{想管理好程序员,只要给他想做的事情并相信他可以。}
\vspace{2mm} \par
\textbf{现在我只期望~云原生开发~的岗位。下面请让我好好的介绍自己。}
%%%%%%%%%%%%%%%%%%%%%%%%%%%%%%%%%%%%%%%%%%%%%%%%%%%%%%%%%%%%%%%%%%%%%%%%%%%%%%%%

%%%%%%%%%%%%%%%%%%%%%%%%%%%%%%%%%%%%%%%%%%%%%%%%%%%%%%%%%%%%%%%%%%%%%%%%%%%%%%%%
% 基本信息
\section{\faInfo\ 基本信息} \vspace{1mm}

{
	\phone \sffamily{手机:}\enspace{(+86) 159-3202-7725}
	\vspace{2mm} \par
	\email \sffamily{邮箱:}\enspace{lizebang.china@gmail.com}
	\vspace{2mm} \par
	\github \sffamily{GitHub:}{https://github.com/lizebang}
	\vspace{2mm} \par
	\faLink \sffamily{博客:}\enspace{https://blog.lizebang.top}
}
%%%%%%%%%%%%%%%%%%%%%%%%%%%%%%%%%%%%%%%%%%%%%%%%%%%%%%%%%%%%%%%%%%%%%%%%%%%%%%%%

%%%%%%%%%%%%%%%%%%%%%%%%%%%%%%%%%%%%%%%%%%%%%%%%%%%%%%%%%%%%%%%%%%%%%%%%%%%%%%%%
% 教育经历
\section{\faGraduationCap\ 教育经历} \vspace{1mm}

\datedline{华北电⼒大学(保定)\enspace 计算机系 \enspace  网络工程专业 \enspace 本科}
{2016 -- 2020}
%%%%%%%%%%%%%%%%%%%%%%%%%%%%%%%%%%%%%%%%%%%%%%%%%%%%%%%%%%%%%%%%%%%%%%%%%%%%%%%%

%%%%%%%%%%%%%%%%%%%%%%%%%%%%%%%%%%%%%%%%%%%%%%%%%%%%%%%%%%%%%%%%%%%%%%%%%%%%%%%%
% 荣誉成就
% \section{\faTrophy\ 荣誉成就}
% \datedline{\textit{荣誉},地点,国家}{时间}
% \datedline{荣誉}{时间}
%%%%%%%%%%%%%%%%%%%%%%%%%%%%%%%%%%%%%%%%%%%%%%%%%%%%%%%%%%%%%%%%%%%%%%%%%%%%%%%%

%%%%%%%%%%%%%%%%%%%%%%%%%%%%%%%%%%%%%%%%%%%%%%%%%%%%%%%%%%%%%%%%%%%%%%%%%%%%%%%%
% 兴趣爱好
\section{\faSunO\ 兴趣爱好} \vspace{1mm}
%%%%%%%%%%%%%%%%%%%%%%%%%%%%%%%%%%%%%%%%%%%%%%%%%%%%%%%%%%%%%%%%%%%%%%%%%%%%%%%%

\begin{itemize}[parsep=1ex]
	\item \faFilm\ {看电影}
	\item \faHeadphones\ {听音乐}
	\item \faCamera\ {摄影}
\end{itemize}
%%%%%%%%%%%%%%%%%%%%%%%%%%%%%%%%%%%%%%%%%%%%%%%%%%%%%%%%%%%%%%%%%%%%%%%%%%%%%%%%

%%%%%%%%%%%%%%%%%%%%%%%%%%%%%%%%%%%%%%%%%%%%%%%%%%%%%%%%%%%%%%%%%%%%%%%%%%%%%%%%
% 个人特征
\section{\faUser\ 个人特征} \vspace{1mm}

\begin{itemize}[parsep=1ex]
	\item 热爱开源
	\item 有责任心
	\item 有团队合作精神
\end{itemize}
%%%%%%%%%%%%%%%%%%%%%%%%%%%%%%%%%%%%%%%%%%%%%%%%%%%%%%%%%%%%%%%%%%%%%%%%%%%%%%%%

%%%%%%%%%%%%%%%%%%%%%%%%%%%%%%%%%%%%%%%%%%%%%%%%%%%%%%%%%%%%%%%%%%%%%%%%%%%%%%%%
% IT 技能
\section{\faCogs\ IT 技能} \vspace{1mm}

\begin{itemize}[parsep=1ex]
	\item 主要编程语言 Golang,但不介意使用其他语言
	\item 熟练使用 git 与 GitHub 对代码进行管理
	\item 使用过 gRPC Ecosystem,主要是 grpc-gateway
	\item 熟悉 Kubernetes 设计构架,了解 Istio 原理
	\item 熟悉 Docker 基本操作,会将项目打包 Docker 镜像
	\item 熟悉计算机网络,了解 TCP/IP、HTTP 等网络协议
	\item 熟悉使用多种 Web 框架,能自己实现简单的 HTTP 服务端框架
	\item 熟练使用 MySQL、MongoDB,使用过 CockroachDB
	\item 熟悉 Linux 下基本的命令行工具,会写 Shell Script
\end{itemize}
%%%%%%%%%%%%%%%%%%%%%%%%%%%%%%%%%%%%%%%%%%%%%%%%%%%%%%%%%%%%%%%%%%%%%%%%%%%%%%%%

%%%%%%%%%%%%%%%%%%%%%%%%%%%%%%%%%%%%%%%%%%%%%%%%%%%%%%%%%%%%%%%%%%%%%%%%%%%%%%%%
% 个人经历
\section{\faUsers\ 个人经历} \vspace{1mm}

% Kubernetes 的学习
\datedsubsection{\textbf{Kubernetes 的学习}}{2017.12 -- 至今}

\vspace{1mm}\par
\noindent
介绍:
\vspace{1mm}\par
\setlength{\parindent}{2ex}
尝试过 minikube,使用过二进制文件部署,现在使用 kubeadm 管理集群。\vspace{1mm}\par
两次长段时间看源码。第一次由于准备与能力不足收获没有想象中那么大,主要提高了编程上的技巧,使代码结构层次清晰易于维护。第二次也就是最近开始的,主要是在看 etcd、kubeadm 以及一些周边库,并学会使用 processon 绘制了一些流程图帮助理解源码。\vspace{1mm}\par
在两次阅读源码之间,学习过设计模式,并阅读翻译了部分 Kubernetes 的文档。在阅读过程中,学会使用 Kubernetes,编写一些例子测试文档中描述的情形。\vspace{1mm}\par
曾为 kubernetes-sigs、AliyunContainerService 等组织提交过 fix 和 feature 类型的 PR。\vspace{1mm}\par

\vspace{2mm}\par
\noindent
收获:
\begin{itemize}[parsep=1mm]
	\item 熟练使用 Kubernetes,了解 Kubernetes 以及 etcd 的工作原理、设计构架。
	\item 提高了编程上的技巧,提高了对问题的思考能力。
\end{itemize}

% Github Detector
\datedsubsection{\textbf{Github Detector}}{2018.07 -- 2018.09}
\faLink \sffamily{地址:}\enspace{https://github.com/TechCatsLab/github-detector}

\vspace{1mm}\par
\noindent
介绍:

\vspace{1mm}\par
\setlength{\parindent}{2ex}
Github Detector 提供了搜索某个 Golang 库被那些库使用过的功能,主要通过整合 GitHub 上现有的 Golang 库的 vendor 依赖文件来实现。\par
核心结构上,拆分成了 Searcher、Downloader、Filter、Codecs、Storer 五个部分。由 Searcher 获取到库的依赖文件路径,再经 Downloader 下载文件,由 Filter 分析是否可以解析并分发给相应的转码器,Codecs 进行转码并将结果送回控制器进行整合,最后由 Storer 存储信息。\par
整个项目过程中,由于需要频繁调用 GitHub API,所以需要考虑限流的问题。起初考虑每个 GitHub Client 维护一个可用次数和恢复时间。每次请求减一次可用次数,用完放入等待队列。思考许久,最终选择使用令牌桶算法进行限流,因为令牌桶算法既限制了请求次数,也将请求均匀分摊到整个时间段。

\vspace{1mm}\par
\noindent

收获:
\begin{itemize}[parsep=1ex]
	\item 对写出好的优雅的代码有一定的体会
	\item 增强自己对数据结构、代码结构、项目框架的理解与设计能力
\end{itemize}

% 商城后台
\datedsubsection{\textbf{Shop API}}{2017.07.18 -- 2017.08.14}

\vspace{1mm} \par
\noindent
介绍:

\vspace{1mm}\par
\setlength{\parindent}{2ex}
这是我编程之路的第一个项目,它是一个使用 echo 框架、采用 MySQL 作为数据库的商城管理后台。了解到一些概念,例如,router、middleware、CORS、Session、Token 等等。

\vspace{1mm}\par
\noindent
收获:

\begin{itemize}[parsep=1ex]
	\item 参与数据库的设计与数据库的重构,了解整个商城后台的设计框架。
	\item 设计并实现了用户地址管理系统、订单交易系统,以及重构了用户系统的部分代码。
	\item 参与了整个商城后台的测试以及与前端的联调测试。
\end{itemize} \vspace{1mm}
%%%%%%%%%%%%%%%%%%%%%%%%%%%%%%%%%%%%%%%%%%%%%%%%%%%%%%%%%%%%%%%%%%%%%%%%%%%%%%%%

\end{document}
